% В этом документе преамбула
\documentclass[a4paper, 12pt]{article}
%%% Работа с русским языком
\usepackage{cmap}					% поиск в PDF
\usepackage{mathtext} 				% русские буквы в формулах
\usepackage[T2A]{fontenc}			% кодировка
\usepackage[utf8]{inputenc}			% кодировка исходного текста
\usepackage[english,russian]{babel}	% локализация и переносы
\usepackage{indentfirst}
\frenchspacing

\renewcommand{\epsilon}{\ensuremath{\varepsilon}}
\renewcommand{\phi}{\ensuremath{\varphi}}
\renewcommand{\kappa}{\ensuremath{\varkappa}}
\renewcommand{\le}{\ensuremath{\leqslant}}
\renewcommand{\leq}{\ensuremath{\leqslant}}
\renewcommand{\ge}{\ensuremath{\geqslant}}
\renewcommand{\geq}{\ensuremath{\geqslant}}
\renewcommand{\emptyset}{\varnothing}

%%% Дополнительная работа с математикой
\usepackage{amsmath,amsfonts,amssymb,amsthm,mathtools} % AMS
\usepackage{icomma} % "Умная" запятая: \$0,2\$ --- число, \$0, 2\$ --- перечисление
\newcommand*{\mathcolor}{}
\def\mathcolor#1#{\mathcoloraux{#1}}
\newcommand*{\mathcoloraux}[3]{%
	\protect\leavevmode
	\begingroup
	\color#1{#2}#3%
	\endgroup
}% Разукрашивание математических символов

%% Номера формул
%\mathtoolsset{showonlyrefs=true} % Показывать номера только у тех формул, на которые есть \eqref{} в тексте.
%\usepackage{leqno} % Нумереация формул слева

%% Свои команды
\DeclareMathOperator{\sgn}{\mathop{sgn}}

%% Перенос знаков в формулах (по Львовскому)
\newcommand*{\hm}[1]{#1\nobreak\discretionary{}
	{\hbox{\$\mathsurround=0pt #1\$}}{}}

%%% Работа с картинками
\usepackage{graphicx}  % Для вставки рисунков
\graphicspath{{images/}{images3/}}  % папки с картинками
\setlength\fboxsep{3pt} % Отступ рамки \fbox{} от рисунка
\setlength\fboxrule{1pt} % Толщина линий рамки \fbox{}
\usepackage{wrapfig} % Обтекание рисунков текстом
\usepackage{caption} %  Чтобы изменять : на .
\captionsetup{labelsep = none}
%%% Работа с таблицами
\usepackage{array,tabularx,tabulary,booktabs} % Дополнительная работа с таблицами
\usepackage{longtable}  % Длинные таблицы
\usepackage{multirow} % Слияние строк в таблице

%%% Теоремы
\theoremstyle{plain} % Это стиль по умолчанию, его можно не переопределять.
\newtheorem{theorem}{Теорема}[section]
\newtheorem{proposition}[theorem]{Утверждение}

\theoremstyle{definition} % "Определение"
\newtheorem{corollary}{Следствие}[theorem]
\newtheorem{problem}{Задача}[section]

\theoremstyle{remark} % "Примечание"
\newtheorem*{nonum}{Решение}

%%% Программирование
\usepackage{etoolbox} % логические операторы

%%% Страница
\usepackage{extsizes} % Возможность сделать 14-й шрифт
\usepackage{geometry} % Простой способ задавать поля
\geometry{top=22.5mm}
\geometry{bottom=22.5mm}
\geometry{left=35mm}
\geometry{right=10mm}
%
%\usepackage{fancyhdr} % Колонтитулы
% 	\pagestyle{fancy}
%\renewcommand{\headrulewidth}{0pt}  % Толщина линейки, отчеркивающей верхний колонтитул
% 	\lfoot{Нижний левый}
% 	\rfoot{Нижний правый}
% 	\rhead{Верхний правый}
% 	\chead{Верхний в центре}
% 	\lhead{Верхний левый}
%	\cfoot{Нижний в центре} % По умолчанию здесь номер страницы

\usepackage{setspace} % Интерлиньяж
\onehalfspacing % Интерлиньяж 1.5
%\doublespacing % Интерлиньяж 2
%\singlespacing % Интерлиньяж 1

\usepackage{lastpage} % Узнать, сколько всего страниц в документе.

\usepackage{soul} % Модификаторы начертания

\usepackage{hyperref}
\usepackage{bookmark}
\usepackage[usenames,dvipsnames,svgnames,table,rgb]{xcolor}
\hypersetup{				% Гиперссылки
	unicode=true,           % русские буквы в раздела PDF
	pdftitle={Заголовок},   % Заголовок
	pdfauthor={Автор},      % Автор
	pdfsubject={Тема},      % Тема
	pdfcreator={Создатель}, % Создатель
	pdfproducer={Производитель}, % Производитель
	pdfkeywords={keyword1} {key2} {key3}, % Ключевые слова
	colorlinks=true,       	% false: ссылки в рамках; true: цветные ссылки
	linkcolor= black,          % внутренние ссылки
	citecolor=black,        % на библиографию
	filecolor=magenta,      % на файлы
	urlcolor=blue           % на URL
}
\usepackage{csquotes} % Еще инструменты для ссылок

\usepackage[style = apa, maxcitenames=2,backend=biber,sorting=nty,bibencoding=utf8]{biblatex}
\addbibresource{bib_curs.bib}
\DeclareLanguageMapping{russian}{american-apa}
\DeclareLanguageMapping{english}{american-apa}

\usepackage{multicol} % Несколько колонок

\usepackage{tikz} % Работа с графикой
\usepackage{pgfplots}
\usepackage{pgfplotstable}

\usepackage{url}
\renewcommand{\thesection}{\hspace*{-1.0em}}
\renewcommand{\refname}{Литература}
\renewcommand{\thesubsection}{\hspace*{-1.0em}}

%table of contents
\usepackage{tocloft}
\renewcommand{\cftsecleader}{\cftdotfill{\cftdotsep}} %cdots

\begin{document}
	
	\thispagestyle{empty}
	\begin{center}
		\Large{\textbf{Правительство Российской Федерации}}
		\vspace{1ex}
		
		\large{\textbf{Государственное образовательное бюджетное учреждение 
		высшего профессионального образования}}
		\vspace{1ex}
		
		\Large{\textbf{Национальный исследовательский университет \\ <<Высшая школа экономики>>}}
	\end{center}
	\vspace{7ex}
	\large{\textbf{Факультет экономических наук}}
	
	\large{ \noindent \textbf{Кафедра теоретической экономики}}
	\vspace{10ex}
	
	\begin{center}
		\textbf{КУРСОВАЯ РАБОТА}
		
		На тему: <<Анализ причин водных конфликтов>>
	\vspace{8ex}
	\end{center}
	
	\begin{flushright}
		\noindent
		\textbf{Студент группы БЭК 142}
		
		\noindent
		Герман Никита Евгеньевич
		\vspace{2ex}
		
		\noindent
		\textbf{Научный руководитель}
		
		\noindent
		Шилова Надежда Викторовна, старший преподаватель
	\end{flushright}
	
	\begin{center}
		\vfill
		Москва 2016
	\end{center}
	
	\newpage
	\tableofcontents
	\newpage
\section{Введение}
В течение последних лет глобальный спрос на пресную воду постоянно увеличивается по причине роста численности населения и увеличения продовольственных потребностей. В связи с этим в некоторых регионах мира удовлетворение спроса за счет собственных подземных источников и поверхностных вод не представляется возможным. Так, например, на Ближнем Востоке и в Северной Африке водных ресурсов регионов недостаточно для обеспечения нужд домохозяйств и фирм, и несмотря на то, что часть пресной воды импортируется или производится в этих регионах, например, методом опреснения морской воды, дефицит пресной воды приводит к ужесточению конкуренции за водные ресурсы как между агентами внутри страны, так и между отдельными странами.

Усиление конкуренции за пресную воду может привести к развитию \textbf{водных конфликтов}, то есть конфликтов, для которых водные ресурсы или водные системы стали одним из основных источников разногласий и споров в контексте экономического и социального развития между отдельными национальными или субнациональными группами населения [\cite{gleick2014water}]. 
Важно отметить, что невозможно дать точную описательную характеристику водным конфликтам. Также моя работа, как и другие исследования в данной области, опираются по большей части на качество эмпирики, ведь люди, занимающиеся исследованием водных конфликтов, имеют базу лишь из \textbf{зарегистрированных водных конфликтов}. На данный момент в подавляющем большинстве случаев могут быть замечены лишь существенные водные конфликты, которые на самом низком уровне возникают между отдельными группами населения страны. Информация о зарегистрированных водных конфликтах позволяют нам таким образом работать с регионами, где произошли наиболее значимые конфликты: межгрупповые и межстрановые.

Статистические данные подтверждают увеличение числа водных конфликтов за последнее время. Так, исследователи The Pacific Institute зарегистрировали более 100 столкновений, произошедших в течение последних 15 лет, в то время как за всю историю ими было зарегистрировано не более 350 водных конфликтов.\footnote{Данные The Pacific Institute \url{http://www2.worldwater.org/conflict/list/}}

Увеличение числа зарегистрированных водных конфликтов не осталось без внимания в научной литературе. Существуют исследования, например, [\cite{mosello}], которые разбирают опыт уже произошедших водных конфликтов в отдельных регионах и выдвигают гипотезы относительно причин возникновения противоречий вокруг водных ресурсов, порой различающиеся в различные временные срезы для одной и той же страны. Также существуют работы, например, [\cite{wolf2005}], которые пытаются аккумулировать опыт каждой страны и выдать некоторый набор причин развития водных конфликтов для различных регионов. Однако не было замечено исследований, которые попытались бы выдать общий для всех стран алгоритм, предсказывающий возникновение водного конфликта в каждой стране на основе имеющихся данных. Также в научной литературе отсутствует механизм ранжирования общих причин возникновения водных конфликтов в порядке их значимости в определении развития водных конфликтов. Отсутствие общей модели возникновения водных конфликтов, базирующейся на реальных данных, сильно сокращает шансы своевременного пресечения конфликтов в регионах ранее \textbf{не подверженных} водным конфликтам, что является важным на фоне усиливающейся конкуренции за водные ресурсы. 

Определим понятие \textbf{подверженности} регионов водным конфликтам. Страна будет называться подверженной водным конфликтам на момент написания работы, если за период активного мониторинга событий командой The Pacific Institute с 1990 по 2016 год в стране было зарегистрировано более 3 водных конфликтов.\footnote{Данные The Pacific Institute \url{http://www2.worldwater.org/conflict/list/}} Соответственно, подверженность страны водным конфликтам отвечает на вопрос, будут ли в стране при прочих равных отмечены водные конфликты в течение ближайшего долгосрочного периода (десятилетий). Таким образом, можно отказаться от точечных оценок, предсказывающих наступит или не наступит водный конфликт в течение одного года, ввиду сильно выраженного случайного характера подобной случайной величины на узком временном горизонте.

Цель данной курсовой работы --- установить общую для всех стран комбинацию внутренних факторов, хорошо предсказывающую подверженность или отсутствие подверженности регионов значимым водным конфликтам, а также выяснить, какие из этих факторов являются наиболее важными в определении будет ли тот или иной регион подвержен водным конфликтам или нет. Также необходимо определить, в каких странах, до сегодняшнего дня не подверженных водным конфликтам, в ближайшем будущем такие конфликты могут возникнуть. 

Для достижения поставленных целей необходимо, во-первых, рассмотреть основные регионы, в которых в течение 1990 - 2016 происходили водные конфликты. Опираясь на предыдущие исследования, например [\cite{ashton}], соберем и обсудим причины зарождения противоречий вокруг водных ресурсов в том или ином регионе. После их выявления необходимо найти данные, связанные с каждой причиной, по как можно большему числу стран мира. Важно отметить, что, так как мы рассматриваем именно подверженность той или иной страны водным конфликтам, то данные должны быть аппроксимированы таким образом, чтобы хорошо описывалось значение того или иного фактора для каждой страны в течение рассматриваемого периода (1990 – 2016). Для поиска значений факторов в работе была основном использована база данных Aqustat.\footnote{Данные из базы Aquastat: \url{http://www.fao.org/nr/water/aquastat/data/query/index.html}} 

На основе имеющихся данных будут применены различные алгоритмы машинного обучения. Их задача - хорошо предсказывать неизвестные алгоритмам ответы для различных объектов. В нашем случае объектами выступают страны, а в качестве ответов следующие бинарные значение: страна подвержена водным конфликтам или нет. Скрывая от алгоритма ответы для некоторой части объектов, мы можем проследить насколько хорошо предсказывает алгоритм ответы, отобрав наилучший из них с точки зрения выбранной системы оценивания.. Построенный классификатор позволит отобрать наиболее значимые факторы, определяющие подверженность существенным водным конфликтам всех регионов мира, а также поможет получить другой алгоритм, который предсказывает, какие страны до этого не подверженные водным конфликтам, в ближайшем будущем могут оказаться ими подверженными.  

Анализ общих для всех стран причин водных конфликтов может помочь некоторым странам, которые на данный момент не подвержены водным конфликтам, предупредить их развитие в будущем. А приведенное в работе ранжирование причин развития водных противоречий в порядке их значимости будет полезно властям, желающим предупредить развитие водных конфликтов, так как позволит им сфокусироваться на решении наиболее значимых проблем, связанных с водными ресурсами.

\section[Ближний Восток]{Ближний Восток\footnote{В страны Ближнего Востока мы не включаем Египет и Иран}}
\subsection{Проблема физического дефицита водных ресурсов на Ближнем Востоке
}
Начнем исследование возможных причин развития водных конфликтов с рассмотрения ситуации на Ближнем Востоке. На сегодняшний день Ближний Восток остается одним из самых засушливых регионов на планете. Так, по данным Aquastat, предоставляемым Food and Agriculture Orgranization of United Nation (далее FAO), несмотря на то, что в 2013 году в этом регионе проживало более 4\% населения Земли, страны Ближнего Востока располагают 1\% от общего количества возобновляемых пресных водных ресурсов Земли (1)\footnote{Везде далее ссылки на расчеты, представленные в приложении, будут отображаться в следующем формате: (k), где k - идентифицирующий номер вычислений в разделе приложения <<Расчеты>>}. В работе [\cite{falkenmark}] был предложен общепринятый индекс, по которому можно измерять степень дефицита водных ресурсов во всех странах. Индекс базируется на среднегодовом объеме доступных возобновляемых водных ресурсов на одного человека [\cite{brown2011}].

\begin{table}[]
\centering
\caption{.\, Индекс Falkenmark}
\label{falkenmark_table}
\begin{tabular}{|c|c|}
	\hline
	\textbf{Индекс Falkenmark ($m^3$ на душу населения)} & \textbf{Категория}         \\ \hline
	\textgreater1700                                          & Нет стресса        \\ \hline
	1000 - 1700                                               & Стресс             \\ \hline
	500 - 1000                                                & Дефицит            \\ \hline
	\textless500                                              & Абсолютный дефицит \\ \hline
\end{tabular}
\end{table}
По данным Aquastat, для таких стран, как Сирия, Израиль, Палестинское государство, для стран Аравийского полуострова на 2013 год характерен дефицит или абсолютный дефицит водных ресурсов, согласно табл. \ref{falkenmark_table} (2). Соответственно, спрос агентов на пресную воду в этих странах скорее всего не может быть удовлетворен за счет собственных внутренних источников. Действительно, по данным [\cite{stategic_middle}] на 2010 год совокупный спрос на пресную воду, например, в Израиле составлял 288 $m^3$ на одного человека, тогда как по данным Aquastat, объем внутренних располагаемых водных ресурсов составлял $241 m^3$. Для Сирии эти же цифры составляли 1055 и 940 $m^3$ соответственно [\cite{stategic_middle}].

Проблема дефицита собственной первичной воды может быть решена различными путями за счет качественно проводимой внутренней политики, например, за счет инвестиций в производство вторичной воды. Однако отсутствие адекватного решения может привести к развитию водных конфликтов. По данным Strategic Foresight Group, за счет вторичной воды, в Сирии удовлетворяется лишь 3\% совокупного спроса, в то время как Израиль на 2010 год обеспечивал около 60\% совокупного спроса сельского хозяйства на пресную воду, а также 35\% спроса промышленного сектора за счет вторичной воды [\cite{stategic_middle}].

Также можно оказывать нужное воздействие на совокупный спрос на пресную воду за счет грамотного формирования тарифов на водоснабжение, меняя структуру потребления водных ресурсов. Например, как известно, основным потребителем пресной воды во всем мире остается сельскохозяйственный сектор, на который приходится более 70\% совокупного спроса на пресную воду, в то время как на Ближнем Востоке этот показатель составляет 75\% (3). Согласно данным Aquastat за 1990 – 2013, p-value для гипотезы о независимости признаков <<среднегодовой общий объем водных ресурсов, использованных на территории данной страны за 1990 – 2013>> и <<средняя за 1990 – 2013 доля воды, использованная сельскохозяйственным сектором от общего объема потребленных водных ресурсов>> составляет примерно $0,01$, причем корреляция между признаками положительная (4). Таким образом, возможно, ограничение на использование водных ресурсов сельским хозяйством в странах, страдающих от дефицита пресной воды, может стать хорошим решением проблемы нехватки воды за счет снижения совокупного спроса. Однако в странах Ближнего Востока скорее всего действует сильное аграрное лобби, которое сохраняет за собой льготные тарифы на водоснабжение, используя большую часть возобновляемых водных источников страны. 

Высокий предъявляемый спрос на воду сельскохозяйственным сектором, низкая доля водных ресурсов, полученных из альтернативных источников и другие проблемы, с которыми сталкиваются страны Ближнего Востока, не дают странам возможности решить самостоятельно проблему естественного дефицита водных ресурсов. Недостаток водных ресурсов, а также большое количество стран, разделяющих один и тот же водный бассейн могли привести к тому, что некоторые страны в регионе оказались подверженными водным противоречиям. 

Стоит однако отметить, что, как показывает пример Ближнего Востока, дефицит воды не является достаточным условием для возникновения водных конфликтов: для крайне засушливых стран Аравийского полуострова не были отмечены водные конфликты за 1990 - 2016. Поэтому необходимо рассмотреть и другие причины развития водных конфликтов в регионе, не связанные напрямую с объемом располагаемых водных ресурсов.

\subsection{Водные конфликты на Ближнем Востоке}
Бассейн реки Иордан стал предметом особо острого конфликта между странами региона после формирования Израиля в 1948 году. Вскоре после этого оформился круг антагонистически настроенных по отношению друг к другу стран, борющихся в том числе и за воды Иорданского бассейна: Израиль, Иордания, Палестинское государство, Сирия и Ливан. Наиболее значимым событием, которое изменило размещение водных ресурсов Иордана между этими странами, стала Арабо-Израильская война 1967 года. Победа Израиля обеспечила за победителем большой объем дополнительно доступных водных ресурсов в основном за счет Голанских высот. На 1993 год около 40\% подземных вод, используемых Израилем, а также около 33\% от среднегодового предложения воды поступало из источников, захваченных в ходе войны 1967 года [\cite{gleick1993}].
 
Конфликт 1967 года стал последним конфликтом, который существенно повлиял на размещение водных ресурсов между странами региона Иорданского бассейна посредством изменения границ государств. В последнее время за счет усиления кооперации между странами при посредничестве международных организаций удалось достигнуть некоторого компромисса по поводу размещения водных ресурсов в регионе. Так, шесть встреч Working Group on Water Resources, на которых присутствовали представители Израиля и Иордании подготовили почву для соглашений на политическом уровне между этими странами по поводу водных ресурсов. Израильско-иорданский мирный договор в 1994 году частично урегулировал претензии обоих государств относительно размещения водных ресурсов. Все это подготовило почву для дальнейшей кооперации между этими двумя странами в области водных ресурсов, самым значимым результатом которой, пожалуй, стал договор о строительстве Канала Двух морей, подписанный в 2005 году. В результате кооперации в период с 1990 – 2015 год, по данным The Pacific Institute, не было зарегистрировано ни одного водного конфликта между Израилем и Иорданией.

В период с 1990 по 2015 год The Pacific Institute зарегистрировал 5 конфликтов, связанных с водными ресурсами, произошедших в Палестинском государстве. Однако эти конфликты произошли не на почве внутренних или межстрановых разногласий относительно распределения воды --- водные ресурсы Палестинского государства использовались в качестве военных мишеней другими государствами. В связи с этим мы не можем классифицировать Палестинское государство, как страну, подверженную на сегодняшний день водным конфликтам.

В течение уже более 5 лет в Сирии идет гражданская война. В работе [\cite{gleick2014}] обсуждается, какую роль могли сыграть водные ресурсы в эскалации внутреннего конфликта в Сирии. По данным Aquastat, на 2011 год в среднем около 89\% всей потребляемой воды  в Сирии приходилось на сельскохозяйственный сектор (5). В связи с высокой зависимостью аграрного сектора от водных ресурсов, из-за сильной засухи в 2006 - 2009 годах более 1,3 миллионов жителей восточной Сирии пострадали из-за потерь в аграрном секторе [\cite{gleick2014}]. Засуха усугублялась серией неправильно принятых решений правительством, связанных с управлением водными ресурсами. В частности в работе [\cite{gleick2014}] отмечается, что правительство не инвестировало достаточно средств в развитие современной ирригационной инфраструктуры, что влекло за собой большие потери воды при орошении и большую нагрузку на имеющиеся водные ресурсы. Таким образом внутренние проблемы, накопившиеся вокруг водных ресурсов, внесли свой вклад в нестабильность в регионе. Поэтому мы классифицируем Сирию как страну, подверженную на данный момент водным конфликтам. 

Полноводность бассейна рек Тигр и Евфрат обеспечивает Ирак и Турцию достаточным количеством водных ресурсов для удовлетворения внутреннего спроса. По данным Aquastat, одному жителю Ирака в обычный год (когда уровень воды в реках страны – средний за 1970 – 2015) доступно около 2500 $m^3$ возобновляемой пресной воды (6). Достаточное количество водных ресурсов в Турции и в Ираке, а также отсутствие сильных климатических шоков обеспечило отсутствие значимых водных конфликтов в этих странах в период с 1990 по 2016 год. Однако стоит опять отметить, что обеспеченность региона водными ресурсами в обычный год не является ни необходимым, ни достаточным условием для того, чтобы водных конфликтов в регионе не возникло. Можно упомянуть опыт строительства дамбы Ататюрка, начатое в 1983 Турцией. Возможность контроля уровня воды Тигра и Евфрата Турцией вызывал серьезные опасения Сирии и Ирака, что привело к нескольким водным конфликтам между этими странами, которые удалось урегулировать мирным путем. 

Подводя итоги рассмотрения случаев возникновения водных конфликтов на Ближнем Востоке, необходимо отметить, что число зарегистрированных водных конфликтов в этом регионе с 1990 по 2015 год уменьшилось по сравнению с предыдущими периодами. Это было достигнуто в основном за счет улучшившейся кооперации вокруг водных ресурсов между странами. Однако вместе с тем мы увидели, что высокий спрос на воду со стороны сельского хозяйства, климатические факторы и обеспеченность региона водными ресурсами являются внутренними факторами, которые могут влиять на подверженность отдельных стран региона водным конфликтам.

\section{Страны бассейна реки Нил}
\subsection{Проблема физического дефицита водных ресурсов в бассейне реки Нил}
Река Нил является одной из важнейших рек Африканского континента. В таких странах как Руанда, Бурунди, Танзания, Кения, Демократическая Республика Конго, Уганда, Эритрея, Эфиопия, Южный Судан, Судан, Египет и Сомали обеспеченность водными ресурсами в той или иной степени зависит от бассейна реки Нил. Попробуем разобраться с проблемой водных конфликтов в регионе с помощью того подхода, который был описан для Ближнего Востока: выясним, сколько в среднем возобновляемых пресных первичных водных ресурсов приходится на одного человека на 2013 год, убирая все сезонные колебания Нила. 

\begin{figure}
	\centering
	\includegraphics[width = \linewidth] {africa}
	\caption{} \label{pic:africa}
\end{figure}

Как можно видеть из рис. \ref{pic:africa}, дефицит водных ресурсов, исходя из табл. \ref{falkenmark_table}, характерен для Египта, Кении и Судана. Согласно данным The Pacific Institute, эти страны и оказались подвержены водным конфликтам: в Кении было зарегистрировано 10 конфликтов, в Судане и Египте по 4 в период с 1990 – 2015 год. Однако также Эфиопия и Сомали, в которых достаточно водных ресурсов, оказались подвержены водным конфликтам: в них было зарегистрировано 4 и 9 конфликтов соответственно. 

Для рассматриваемого региона стоит отметить проблему острой неопределенности относительно объема располагаемых водных ресурсов в следующем периоде. Так, по данным [\cite{stategic_nile}], ежегодный объем воды в Ниле колебался в 1965 – 2010 годах между 47 и 117 млрд $м^3$. Неопределенность снижает шансы на успешную кооперацию в регионе, что могло также стать причиной такого большого числа водных конфликтов. Также в работе [\cite{ashton}] высказывается предположение о том, что конкуренция между странами Африки, лежащими в верхнем и нижнем течении одного и того же водного источника, является основной причиной водных конфликтов в регионе. Действительно, согласно базе данных Aquastat, отношение среднего за 1990 – 2015 объема водных ресурсов, который поступает из-за границы, к общему объему располагаемых возобновляемых водных ресурсов внутри страны за год (dependency ratio), для Египта составляет 97\%, для Судана 96\%, для Сомали 59\% при среднемировом значении в 22\% (8).  Рассмотрим теперь другие причины развития водных конфликтов в этом регионе на исторических примерах.

\subsection{Водные конфликты в Северной Африке}
В работе [\cite{wolf2005}] было отмечено, что водные конфликты чаще встречаются в странах, в которых сказалось влияние резкого воздействия на распределение водных ресурсов, в частности посредством строительства обширной плотины или изменения ирригационной схемы. В 2011 году Эфиопия объявила о строительстве <<Великой Эфиопской Плотины Возрождения>>. Эта плотина, использующая воды Голубого Нила, должна стать самой большой гидроэлектростанцией в Африке, имея возможность оказывать влияние на уровень воды всего бассейна реки Нил. Как уже было отмечено выше, у таких стран, как Египет и Судан высокий dependency ratio. Река Нил, являющаяся основным источником воды в этих странах, берет свои истоки в том числе и в Эфиопии. Обеспокоенность властей Египта и Судана по поводу доступного им объема воды в реке Нил после завершения строительства гидроэлектростанции, не повлияло на решение Эфиопских властей построить плотину, в связи с чем между странами возник водный конфликт, который до сих пор не был разрешен --- Египет все еще требует уступок относительно доли в своем потреблении вод Нила.

Качество предоставляемой государством воды также стало одной из причин развития водных конфликтов. Так, по данным the Pacific Institute, в 2012 году жители из египетской провинции Минуфия заблокировали выход из здания больницы министру здравоохранения в знак протеста против качества предоставляемой государством воды населению, которая стала причиной болезни 400 жителей провинции. 

В качестве еще одного не указанного фактора, который мог оказать влияние на развитие водных конфликтов в странах, принадлежащих бассейну реки Нил, можно обозначить степень этнического разнообразия в регионе. В работе [\cite{ashton}] упоминается, что одной из основных проблем Африканских стран на данный момент остаются последствия неправильного оформления границ между странами, не соответствующие национальной общности того или иного региона. Это привело к тому, что обширное этническое разнообразие народов Африки в купе с неправильным оформлением границ, возможно, являются одними из основных причин нестабильности на континенте. Так, по данным The Pacific Institute, в Кении водные конфликты по большей части происходили между агентами внутри страны, которые относились к разным кланам, племенам или деревням, представляющим конфликтующие этнические группы.

Подводя итоги, нам удалось выявить следующие причины развития водных конфликтов, присущие региону, принадлежащему бассейну реки Нил: конкуренция за воду между странами, лежащими в верхнем и нижнем течении одного и того же водного источника, строительство обширных дамб на реках, разделяемых большим количеством стран, качество водных ресурсов, предоставляемых государством населению, а также этническое разнообразие в регионе.

\section{Средняя Азия}
\subsection{Проблема физического дефицита водных ресурсов в Средней Азии}
Поверхностная пресная вода является главным источником питьевой воды и воды, предназначенной для сельского хозяйства в странах Средней Азии. В отличие от рассмотренных выше регионов, согласно данным Aquastat и индексу Falkenmark, ни одна из стран Средней Азии: Кыргызстан, Узбекистан, Таджикистан, Казахстан, Туркменистан --- не подвержена стрессу первичных водных ресурсов на 2013 год (9). Однако стоит отметить, что спрос на водные ресурсы в различных регионах может сильно различаться, в связи с чем индекс Falkenmark отображает не вполне точную картину. Как уже было отмечено ранее, важность сельского хозяйства в хозяйственной деятельности региона определяет высокий спрос на пресную воду. Действительно, во всех странах Средней Азии средняя за 1990 – 2015 доля воды, использованная сельским хозяйством от общего объема потребленных водных ресурсов, составляет более 91\% (10). Высокий спрос на водные ресурсы, предъявляемый сельским хозяйством, объясняется особенностью культур, выращиваемых на территории стран Центральной Азии: риса и хлопка --- требующие обильное орошение.

Проблему нехватки водных ресурсов в регионе из-за высокого спроса можно увидеть на печальном примере Аральского моря, обмеление которого началось после масштабного увеличения площади орошаемых земель в регионе. Это привело к катастрофическому опустыниванию и загрязнению окружающих вод, не говоря уже о последствиях, которые чрезмерное использование водных ресурсов нанесло уникальной биосистеме Аральского моря. 

Дефицит первичных водных ресурсов, скорее всего является причиной конфликтов в регионе, однако, как не раз отмечалось в этой работе, существуют страны, которые, страдая от дефицита пресной воды, не были подвержены водным конфликтам с 1990 по 2016 год. Это достигалось как за счет кооперации между странами региона, на примере Израиля и Иордании, так и за счет граммотной внутренней политики, направленной, например, на увеличение эффективности ирригационных систем и на производство вторичной воды. Следует более подробно разобраться с причинами, которые спровоцировали, согласно данным The Pacific Institute, 9 конфликтов между странами региона с 1990 по 2016. Перейдем к рассмотрению случаев  водных конфликтов в регионе.

\subsection{Водные конфликты в Средней Азии}
После развала Советского Союза в 1991 году страны Средней Азии столкнулись с конфликтующими интересами относительно распределения пресной воды. В 1991 году разногласия между Кыргызтаном, Казахстаном, Узбекистаном, Туркменистаном и Таджикистаном вокруг водных ресурсов в бассейнах рек Сырдарья и Амударья в основном возникли относительно возможности совместного поддержания водной инфраструктуры: плотин и ирригационных систем [\cite{mosello}].

Стоит отметить, что одним из факторов, обусловивших подверженность региона водным конфликтам, стала конкуренция между странами, лежащими в верхнем и нижнем течении одного и того же водного источника. В Средней Азии можно четко определить страны, которые обладают богатыми водными ресурсами и располагаются в верхнем течении крупнейших рек: это Кыргызтан и Таджикистан --- от стран, которые являются зависимыми от политики соседнего государства в отношении водных ресурсов: Узбекистан и Туркменистан. 

Строительство обширных дамб стало причиной нескольких конфликтов в этом регионе. Так, в 1992 году разногласия между Туркменистаном и Узбекистаном разгорелись вокруг Туямуюнского водохранилища, построенного в Туркменистане в 1979 году. Оба государства были недовольны чрезмерным использованием водных ресурсов противоположной стороной. Летом 1999 года, по данным The Pacific Institute, Таджикистан выпустил 700 млн. $m^3$ пресной воды из своего резервуара, не предупредив заранее о своих действиях соседей, находящихся в нижнем течении реки Сырдарья. Это привело к разрушению обширных посевных площадей в низовьях реки. Также был отмечен случай, когда в 2008 году жители нескольких деревень Таджикистана попытались ликвидировать плотину, построенную Кыргызстаном, которая блокировала ветку ирригационного канала, идущую в эти деревни.

Однако не только строительство плотин, проблема <<верхнее-нижнее течение>> и высокий спрос на водные ресурсы стали причинами водных конфликтов в регионе. В работе [\cite{mosello}] отмечается, что в регионе, также, как и в Африке, наблюдается сильное этническое разнообразие, которое усугубляется неправильным оформлением границ стран после развала Советского Союза. По мнению автора, эти факторы могли привести к водным конфликтам между странами Средней Азии. Также отмечается высокий темп роста численности населения в Центральной Азии в качестве фактора, определившего подверженность региона водным конфликтам [\cite{mosello}]. 

Но, возможно, ключевая причина, по которой в этом регионе до сих пор наблюдаются сильные разногласия по поводу распределения водных ресурсов состоит в отсутствии значимой кооперации между странами Средней Азии. Конечно, само отсутствие сотрудничества вокруг водных ресурсов может быть производным от других причин, например, политических разногласий между странами. Однако все же, если вспомнить пример Иордании и Израиля, противоречия между странами Средней Азии кажутся несущественными по сравнению с теми проблемами, которые пришлось преодолеть Израилю и Иордании на пути к совместной кооперации вокруг водных ресурсов. Используя данные Transboundary Water Dispute Database\footnote{Данные Oregon State University:\newline \url{http://transboundarywater.geo.orst.edu/database/interfreshtreatdata.html}} и Aquastat, можно предложить следующий индекс, для измерения степени отсутствия кооперации отдельной страны вокруг водных ресурсов:
\begin{equation}\label{eq:1}
Coop_i  = \sum_{k = 1}^{n}\sum_{l = 1}^{m}\frac{\text{dependancy ratio}_{l, k}}{N_i\cdot\sqrt{n_k}}
\end{equation}
Где $n_k$ – количество соглашений между страной i и другими странами относительно водных ресурсов бассейна k к которому принадлежит страна i.\footnote{Везде далее в работе учитываются лишь соглашения относительно водных ресурсов, датируемые с 1960 года, а не с 1820, данные по которым присутствуют в Transboundary Water Dispute Database. Это сделано в связи с тем, что многие современные страны до 1960 были колониями, так что метрополия вела от их имени переговоры. Также необходим более узкий интервал времени в связи с «износом» силы соглашений.} А $\sum_{l = 1}^{m} \text{dependancy ratio}_l$ --- суммированный по всем странам, принадлежащим бассейну k, dependancy ratio. $N_i$ --- число стран, с которыми страна i имеет общий бассейн. Смысл этого индекса состоит в следующем: чем больше сумма dependency ratio стран, принадлежащих одному бассейну, пронормированная на значимость того или иного бассейна в обеспечении водой региона, тем вероятнее конфронтация в регионе типа «верхнее – нижнее течение реки». Однако, чем больше $n_k$, тем менее остро ощущается эта проблема. Оказывается, что средняя величина введенного коэффициента для бассейна Аральского моря составляет 11 пунктов, а для реки Нил, несмотря на большее число стран в бассейне, --- 8 пунктов (11). 

Подводя итоги рассмотрению опыта водных конфликтов в странах Средней Азии, удалось определить важность таких признаков, как обширное строительство плотин, этническое разнообразие в регионе и уровень кооперации между странами в предопределении подверженности данного региона водным конфликтам.
\section{Другие причины развития водных конфликтов}
Стоит отметить, что вышеперечисленные регионы: Ближний Восток, Северная Африка, Средняя Азия --- не вмещают в себе всю совокупность стран, подверженных водным конфликтам. Однако на их примере нам удалось разобраться с основными причинами водных конфликтов, возникающих в той или иной области земного шара. В Индии, Китае, Иране, Афганистане и Пакистане водные конфликты, как внутренние, так и межстрановые, происходили в основном по причинам, которые уже были отмечено выше. Однако отметим здесь еще несколько причин, которые некоторые авторы предлагают в качестве основных, определявших подверженность любого региона водным конфликтам. 

В работе [\cite{wolf2005}] отмечается загрязненность источников пресной воды в качестве важной причины развития водных конфликтов. По мнению авторов, загрязненность может привести к обострению внутренней обстановки, если она оказывает сильное воздействие на жизнедеятельность людей или на экологическую обстановку региона. Также некоторые авторы отмечают плотность населения в качестве одной из причин, приведшей к эскалации водных конфликтов, например, в таких регионах, как в Индии и Китае.

Важно отметить, что многие страны имеют инструменты, позволяющие им сгладить последствия дефицита внутренних водных ресурсов. При рассмотрении ситуации на Ближнем Востоке было отмечено, что поощрение производства вторичной воды, а также изменение в структуре потребления водных ресурсов могут помочь справиться с последствиями естественного дефицита. Также мы рассматривали возможность кооперации между странами в качестве одного из решения проблемы. Однако отсутствие значимых шагов властей в некоторых странах, направленных на решение проблемы, может свидетельствовать о некачественном функционировании управляющего аппарата страны. В исследовании [\cite{corruption}] описывается как последствия высокой коррумпированности влияют на размещение водных ресурсов между агентами внутри страны. Авторы исследования считают, что водный кризис во многих регионах является следствием некачественного государственного управления, и коррупция стала главным катализатором данного кризиса.

В результате детального рассмотрения опыта водных конфликтов в различных регионах, нам удалось выделить большую часть причин, которые исследователи предлагают в качестве основных, ведущих к эскалации водных конфликтов. Рассмотрим теперь модель, которая поможет разобраться какие из рассмотренных причин являются самыми важными в определении подверженности водным конфликтам большинства стран планеты. Также определим, в каких регионах в настоящий момент не подверженных водным конфликтам, таковые могут в ближайшем будущем возникнуть.

\section{Модель классификации}
\subsection{Описание набора признаков и целевой переменной}
Для достижения основных целей данной работы будут использованы алгоритмы классификации машинного обучения. В классической задаче машинного обучения набору объектов ставится в соответствие некоторая параметрическая функция, называемая алгоритмом. Путем оптимизации функции ошибки для алгоритма на этапе обучения, проходит настройка параметров алгоритма под имеющиеся объекты для выдачи лучшего ответа. Приведем пример: линейная регрессия является одним из алгоритмов машинного обучения. В качестве параметров для данной функции в простейшем случае выступают коэффициенты линейной комбинации признаков. В качестве функции ошибки для данного алгоритма может выступать среднеквадратичная ошибка (МНК), оптимизируя которую, мы получаем коэффициенты весов признаков, которые с точки зрения выбранного семейства параметрических функций и выбранной функции ошибки наилучшим образом описывают линейную зависимость объект ---> ответ. 

Как уже отмечалось, в работе рассматривается именно подверженность региона водным конфликтам, а не задача выяснения произойдет ли в определенном году в некоторой стране водный конфликт или нет. Поэтому классификатор должен получать на входе усредненные данные по признакам за 1990 – 2016 год по некоторым странам и выдавать ответ: подвержены ли они водным конфликтам или нет. В качестве признакового поля в модели будут выступать причины водных конфликтов, которые мы рассмотрели выше. Для того, чтобы обучить алгоритм и строить предсказания относительно динамики целевой переменной, необходимо количественное описание признаков для каждой из стран.

Но для начала еще раз остановимся на целевой переменной. В качестве целевой переменной (ответа) будет выступать следующая бинарная переменная: для страны, которая не была подвержена водным конфликтам ставится в соответствие значение (-1), а для страны, которая была подвержена ставится значение (1). Определимся еще раз, какие страны были подвержены водным конфликтам за рассматриваемый период. Как уже отмечалось выше, мы назовем те страны подверженные водным конфликтам, для которых было зарегистрировано более 3 водных конфликтов с 1990 - 2016 год. Такой порог был выбран не случайно. Во-первых, мы таким образом отделяем страны, для которых водные конфликты стали больше случайностью, чем закономерностью: например, в Марокко за 1990 – 2016 год был зарегистрирован 1 водный конфликт, и в литературе этот опыт почти не рассматривается ввиду отсутствия систематической составляющей этой проблемы в стране. Во-вторых, мы не ожидаем наилучших результатов от построенного классификатора. И если эти ошибки будут выглядеть, например, следующим образом: страну, в которой на самом деле произошло только три водных конфликта за рассматриваемый период, классификатор определил как подверженную конфликтам, --- то эти ошибки можно считать несущественными: алгоритм все равно нашел некоторую закономерность, которая отделяет регионы, в которых иногда происходят водные конфликты от регионов, где они не наблюдаются. Выбранный порог в 3 страны позволяет задать значение целевой переменной для каждой из стран. В итоге, для таких стран, как Китай, Индия, Кения, Сомали, Кыргызтан, Эфиопия, Узбекистан, Таджикистан, Судан, Иран, Казахстан, Египет, Афганистан, Пакистан, значение целевой переменной будет равно (1), а для других стран --- (-1). 

Теперь рассмотрим, как строилось количественное описание причин водных конфликтов для всех стран. В качестве основного источника информации о данных была выбрана база данных Aquastat, предоставляемая Food and Agriculture Organization of United Nation, из которой в итоге были отобраны признаки со значениями с 1990 по 2016 год для 200 стран. Стоит отметить, что база данных предоставляет значения того или иного признака с периодичностью раз в 5 лет. Например, если мы рассматриваем период с 1988 – 1997 год, то в данных будут доступны только два значения: за 1988 – 1992 и 1993 – 1997. В основном значение того или иного признака строится на окончание 5-летнего периода: в нашем случае, это на 1992 и 1997 год соответственно. Перейдем к рассмотрению признаков, использованных при построении модели, каждый из которых связан с той или иной причиной подверженности водным конфликтам, обозначенной в предыдущем разделе.

Начнем с рассмотрения признаков, связанных с объемом водных ресурсов, доступных той или иной стране. В работе [\cite{priznaki}] раскрывается определение некоторых признаков, представленных в базе Aquastat. Объем водных ресурсов в стране описывается в работе, как <<средний объем воды в реках и пополнение в водоносном слое, вызванное естественными осадками>>. Под фактически располагаемыми возобновляемыми водными ресурсами подразумевается сумма внутренних и внешних возобновляемых водных ресурсов. При этом слово <<фактический>> отражает тот факт, что при подсчете значения признака принимается во внимание уменьшение объема воды, поступающего из-за рубежа, возникающее в связи с использованием водных ресурсов странами, находящимися в верхнем течении водного источника. Кроме того, стоит также отметить, что большинство рассматриваемых показателей объема водных ресурсов, предоставляемых Aquastat, являются долгосрочными для стран: они отражают среднегодовые значения признака, полученные исходя из наблюдений, начатых еще в 1993 году. Теперь можно перейти к непосредственному рассмотрению отобранных признаков:\footnote{Везде далее в основной части работы не будет рассмотрено, как технически получались значения для тех или иных признаков или как обрабатывалась база данных Aquastat. Подробно ознакомиться с этим можно в приложении в разделе <<Обработка признаков>>. Отметим, что для обработки некоторых пропусков в данных Aquastat использовались данные UNEP; данные, представленные на сайте ЦРУ: \url{https://www.cia.gov/library/publications/the-world-factbook/fields/2202.html}; данные Knoema: \url{https://knoema.com/atlas}}
\begin{enumerate}
	\item Долгосрочный общий объем располагаемых внутренних водных ресурсов в $m^3$, деленный на среднюю численность населения за 1990 – 2015 год.\footnote{Везде далее <<на душу населения>> означает <<деленный на среднюю численность населения за 1990 – 2015 год>> Под внутренними водными ресурсами подразумеваются подземные и поверхностные водные источники, которые получили свое наполнение через эндогенные осадки страны.}
	\item Долгосрочный фактический объем поверхностных водных ресурсов в $m^3$ на душу населения, поступающий из-за границы и не подчиняющийся никаким специальным договорам, регулирующим объемы водозабора из этих источников.
	\item Долгосрочный фактический объем поверхностных водных ресурсов в $m^3$ на душу населения, протекающий по рекам, формирующим границы между странами.
	\item Долгосрочный фактический объем подземных водных ресурсов в $m^3$ на душу населения, поступающий из-за границы и не подчиняющийся никаким специальным договорам, регулирующим объемы водозабора из этих источников.\footnote{Везде далее для сокращения обозначений будет использована эта нумерация обозначенных признаков. Например: <<1)>> -- Общий объем располагаемых внутренних водных ресурсов, деленный на среднюю численность населения за 1990 – 2015 год}
\end{enumerate}
Введенные признаки имеют непосредственную связь с общим фактическим объемом располагаемых страной водных ресурсов. Этот показатель, согласно FAO, равен 1) + 2) + фактический среднегодовой объем располагаемых водных ресурсов, поступающих из-за границы, закрепленный некоторыми соглашениями - фактический среднегодовой объем водных ресурсов, поступающий за рубеж из данной страны, который закреплен некоторыми соглашениями + 4)\footnote{Наглядная математическая связь между <<водными признаками>> представлена по ссылке: \url{http://www.fao.org/nr/water/aquastat/data/wrs/readPdf.html?f=AFG-WRS_eng.pdf}}. Однако мы не будем рассматривать признаки, которые связаны с некоторыми соглашения, так как учтенные договора в данных Aquastat имеют сильную дипломатическую базу. Hаличие таких соглашений является следствием водных конфликтов между странами, а не их причиной. Поэтому мы не можем учесть эти признаки в нашей модели, так как наша задача --- предсказать развитие водных конфликтов. 

Теперь перейдем к рассмотрению других признаков, отобранных из базы данных Aquastat:
\begin{enumerate}
	\setcounter{enumi}{4}
	\item Нормализованный индикатор, отражающий уровень колебаний в объеме располагаемых водных ресурсов в стране в течение нескольких лет. Считается, как стандартное отклонение годовых за 1950 – 2010 объемов фактически располагаемых водных ресурсов, деленное на средний за 1950 – 2010 объем фактически располагаемых водных ресурсов. Признак нормализован таким образом, чтобы значение <<0>> отражало самый низкий уровень колебаний в объеме располагаемых водных ресурсов за 1950 – 2010, а <<5>> самый высокий уровень среди стран выборки.
	\item Dependancy ratio, который уже был ранее рассмотрен. Отражает долгосрочную долю водных ресурсов, поступающую из-за границы ( 2) + 4) ), от долгосрочного фактического объема располагаемых водных ресурсов страны ( 2) + 4) + 1) ).
	\item Доля воды, использованная сельскохозяйственным сектором, от общего объема потребленных водных ресурсов. Здесь важно отметить, что данные по доле для некоторых стран сильно колебались в разных периодах, что не позволяет одним числом хорошо приблизить значение признака для этих стран. Поэтому для сведения данных по странам за 1990 – 2015 к единой таблице были отобраны значения доли для всех стран, и посчитана для каждой из стран общая дисперсия. Если для некоторого региона дисперсия признака не превышала 0.05, а самих наблюдений было больше одного, то для этой страны было подсчитано среднее по доле за 1990 - 2015.
	\item Средняя за 1990 – 2015 год доля населения от общей числа людей страны, имевшая доступ к воде, прошедшей хотя бы первичную обработку. Таковой можно называть воду, предоставляемую домохозяйствам через общую систему водоснабжения, скважины и т.д.
	\item Номинальный ВВП на душу населения на 2013 год в текущих долларах США.
	\item Среднегодовой темп прироста ВВП с 1990 – 2013 год.
	\item Средняя плотность населения с 1990 – 2013 год.
	\item Изменение общей вместимости дамб в $\text{км}^3$. Для каждой страны считалось как максимальная вместимость дамб внутри страны за 1990 – 2015 за вычетом минимальной вместимости на душу населения
\end{enumerate}
Также были отобраны следующие признаки из других источников:
\begin{enumerate}
	\setcounter{enumi}{12}
	\item Индекс восприятия коррупции на 2015 год\footnote{Данные Transparency International: \url{https://www.transparency.org/cpi2015/}}
	\item Уже рассмотренный на стр. \pageref{eq:1} индекс кооперации, 
	опирающийся на соглашения относительно водных ресурсов, датируемых с 1960 года
	\item Water Quality Index на 2011 год, предложенный в работе [\cite{srebotnjak}], который является индикатором загрязненности пресных водных ресурсов страны (чем выше индекс, тем ниже загрязненность).
\end{enumerate}
В результате, для построения предсказательной модели относительно того, будет ли страна подвержена водным конфликтам, было отобрано 15 признаков, которые связаны с теми причинами, которые уже были выделены ранее в работе.
\subsection{Анализ построенных признаков
}
Рассмотрим список из 103 стран, не имеющих пропусков в данных. Среди них присутствуют все регионы, подверженные водным конфликтам, кроме Афганистана и Сомали, для которых отсутствуют данные сразу по нескольким признакам. Выведем количественное описание полученных признаков. представленное в табл. \ref{tab:priznak}.\footnote{Значения x1, x2, … отражают закодированные признаки 1), 2), … 15)}
\begin{table}
	\centering
	\caption{.\, Описание признаков}
	\label{tab:priznak}
	\begin{tabularx}{\textwidth}{|c|X|X|X|X|}
		\hline
		& среднее       & стандартное отклонение      & максимальное значение       & минимальное значение    \\ \hline
		x1  & 20499.843 & 67330.138 & 573742.828 & 24.034  \\ \hline
		x2  & 2274.330  & 5215.550  & 40308.133  & 0       \\ \hline
		x3  & 721.343   & 3716.518  & 34949.566  & 0       \\ \hline
		x4  & 8.123     & 64.658    & 646.824    & 0       \\ \hline
		x5  & 1.875     & 0.891     & 4.900      & 0.600   \\ \hline
		x6  & 27.889    & 32.309    & 97.00      & 0       \\ \hline
		x7  & 0.570     & 0.318     & 0.978      & 0.003   \\ \hline
		x8  & 0.8449    & 0.172     & 1          & 0.369   \\ \hline
		x9  & 11104.876 & 15732.449 & 75146.667  & 243.183 \\ \hline
		x10 & 0.050     & 0.026     & 0.130      & -0.019  \\ \hline
		x11 & 99.98     & 129.56    & 871.250    & 1.640   \\ \hline
		x12 & 0.000312  & 0.000911  & 0.0056     & 0       \\ \hline
		x13 & 43.417    & 21.174    & 90         & 12      \\ \hline
		x14 & 11.219    & 10.792    & 76.6       & 0       \\ \hline
		x15 & 68.938    & 16.213    & 99.41      & 34      \\ \hline
	\end{tabularx}
\end{table}
Прежде чем приступить к обучению классификатора на основе построенных данных, проанализируем поведение некоторых ключевых признаков, которые связаны с важными причинами возникновения конфликтов, обозначенными в предыдущем разделе. 

Для многих стран, подверженных водным конфликтам, не раз отмечалась свойственная им высокая доля потребляемых водных ресурсов сельскохозяйственным сектором от общего объема потребления (далее эта доля будет обозначаться признаком 7) ).
 \begin{figure}
	 	\centering		
		\includegraphics[width = 0.7\linewidth]{Priznak7}
		\caption{}\label{priznak_7}
\end{figure}
Рассмотрим рис. \ref{priznak_7}. На оси Ох расположены значения целевой переменной. Как уже отмечалось, значение <<1>> соответствует странам, которые были подвержены водным конфликтам, а <<-1>> странам, которые не были подвержены. <<Ящики>> на рис. \ref{priznak_7} отражают те границы, в которые попала большая часть значений (95\%). Синий ящик показывает, куда попали значения признака 7), при условии, что рассматриваются только страны не подверженные водным конфликтам. А зеленый ящик показывает то же самое, только для стран, подверженных водным конфликтам. Можно увидеть очень хорошее выделение: ящик для стран, подверженных водным конфликтам лежит выше синего ящика, что позволяет заключить, что в странах, подверженных водным конфликтам, очень высокое значение признака 7), в то время как для стран, не подверженных конфликтам, такие большие значения не свойственны. 

Также уже отмечалось, что низкий объем располагаемых водных ресурсов не является ни необходимым, ни достаточным условием развития водных конфликтов в той или иной стране. Рассмотрим этот факт на примере признака <<Долгосрочный общий объем располагаемых внутренних водных ресурсов, деленный на среднюю численность населения за 1990 – 2015 год>>, который является признаком 1).
\begin{figure}
	\centering		
	\includegraphics[width = 0.7\linewidth]{Priznak1}
	\caption{}\label{priznak_1}
\end{figure}
Обратимся к рис. \ref{priznak_1}. Он хорошо показывает, что хоть для стран, подверженных водным конфликтам, и характерен относительно низкий объем внутренних располагаемых водных ресурсов, тем не менее есть страны, которые, будучи не подверженными конфликтам, в то же время располагают меньшим объемом водных ресурсов, чем страны, подверженные водным конфликтам. 
\subsection{Обучение классификатора}
На построенном признаковом пространстве обучим несколько классификаторов и проверим их предсказательную способность в определении подвержена ли некоторая страна водным конфликтам. Табл. \ref{confussion} иллюстрирует каким образом могут распределиться ответы классификатора.
\begin{table}[]
	\centering
	\caption{.\, Confussion Matrix}
	\label{confussion}
	\begin{tabular}{|c|c|c|}
		\hline
		& y (Predicted) = 1 & y (Predicted) = -1 \\ \hline
		y (Actual) = 1  & TP                & FN                 \\ \hline
		y (Actual) = -1 & FP                & TN                 \\ \hline
	\end{tabular}
\end{table}
В машинном обучении есть несколько метрик качества, позволяющих определить пригодность того или иного алгоритма. Для оценки качества классификатора будут использованы следующие метрики:
\begin{equation} \label{eq:2}
\begin{aligned}
Recall &= \frac{TP}{TP + FP} \\
Precision &= \frac{TP}{TP + FN}
\end{aligned}
\end{equation}
Recall отражает, насколько широко мы охватываем класс стран, подверженных водным конфликтам. Чем выше значение этой метрики, тем больше мы можем рассчитывать на то, что классификатор найдет все страны, подверженные конфликтам. Accuracy показывает, насколько мы можем доверять предсказаниям классификатора в том случае, если он выдает ответ, что страна подвержена водным конфликта. Чем больше значение accuracy, тем мы можем быть увереннее в том, что алгоритм правильно предсказал, что страна подвержена водным конфликтам.

Для оценки качества модели классификации, а также для настройки гиперпараметров алгоритмов необходимо, скрыв от алгоритма часть стран, обучить его на оставшихся объектах и проверить его качество на отложенной части стран, на которой он не обучался. Для этого используем процесс кросс-валидации. В ходе кросс-валидации вся выборка разбивается на k случайных равных частей, после чего k-1 часть выступает в качестве выборки, на которой проходит обучение алгоритма, а оставшаяся часть используется для проверки его качества. После этого уже другие k-1 частей выборки берутся для обучения и так далее. В итоге получается k алгоритмов, для которых уже подсчитаны значения метрик качества на неизвестных классификатору частях выборки, а также настроены гиперпараметры. Полученные k результатов для метрик усредняются, и выдается ответ об ожидаемом значении метрики качества на неизвестных алгоритму странах. Подобный подход был использован в данной работе. Для проверки качества алгоритмов выборка была разбита 2 раза случайным образом на 5 равных частей, после чего были применены 2 раза схемы кросс-валидации для каждого базового алгоритма.
Все использованные в работе алгоритмы выдают следующий ответ для каждого объекта:
\[
y(x) = \sgn(w \cdot x)
\]
Настраиваемым параметром в ходе обучения выступает вектор весов w. Чем больше абсолютное значение w для стандартизированных признаков, тем важнее тот или иной признак в определении ответа классификатора. В нашей модели это означает, например, что если вес коэффицента для некоторого признака сильно отрицательный, то, чем больше значение этого признака для некоторой страны, тем вероятнее страна окажется не подверженной водным конфликтам. Чем меньше  --- тем вероятнее, наоборот, окажется. 

Для построения классификатора были отобраны следующие алгоритмы: логистическая регрессия (LR), метод опорных векторов (SVC) и случайный лес (RF)\footnote{Случайный лес в качестве ответа выдает feature\_importances\_, который отражают важность того или иного признака в определении ответа. Все feature\_importances $\geqslant$ 0 и в сумме дают единицу}. В табл. \ref{classifiers} приведены результаты, полученные после проведения двух схем кросс-валидации на этих алгоритмах\footnote{Ознакомиться с процессом подбора гиперпараметров для того или иного алгоритма можно в приложении в разделе <<Обучение классификаторов>>}. N отражает количество стран, которое использовались в выборке для обучения алгоритма. Значения в ячейках, например, (x7 – SVC) отражают вес коэффицента w для признака x7, полученный после обучения алгоритма на всей выборке, с настроенными в ходе кросс-валидации гиперпараметрами.
\begin{table}[]
	\centering
	\caption{.\, Результаты классификаторов}
	\label{classifiers}
	\begin{tabular}{|l|l|l|l|l|l|l|}
		\hline
		& LR(1)  & SVC(1) & RF(1) & LR(2)  & SVC(2) & RF(2) \\ \hline
		x1       & -0.107 & -0.176 & 0.059 & -0.151 & -0.650 & 0.092 \\ \hline
		x2       & -0.439 & -0.485 & 0.016 & -0.678 & -0.496 & 0     \\ \hline
		x3       & -0.188 & -0.215 & 0.005 & -0.300 & -0.716 & 0     \\ \hline
		x4       & 0.226  & 0.01   & 0.039 & 0.290  & 0.033  & 0     \\ \hline
		x5       & 0.0036 & 0.123  & 0.030 & -0.025 & 0.160  & 0.033 \\ \hline
		x6       & 0.297  & 0.133  & 0.074 & 0.385  & 0.132  & 0.025 \\ \hline
		x7       & 0.837  & 0.919  & 0.183 & 1.234  & 1.511  & 0.172 \\ \hline
		x8       & -0.064 & -0.08  & 0.045 & -0.049 & -0.070 & 0.087 \\ \hline
		x9       & -0.209 & -0.241 & 0.094 & -0.267 & -0.348 & 0.141 \\ \hline
		x10      & 0.302  & 0.335  & 0.138 & 0.415  & 0.567  & 0.025 \\ \hline
		x11      & -0.171 & -0.330 & 0.050 & -0.304 & -0.440 & 0.025 \\ \hline
		x12      & -0.163 & -0.285 & 0.034 & -0.258 & -0.355 & 0     \\ \hline
		x13      & -0.440 & -0.524 & 0.118 & -0.617 & -0.751 & 0.125 \\ \hline
		x14      & -0.025 & 0.392  & 0.066 & -0.036 & 0.290  & 0.157 \\ \hline
		x15      & 0.309  & 0.363  & 0.049 & 0.513  & 0.622  & 0.119 \\ \hline
		N        & 103    & 103    & 103   & 103    & 103    & 103   \\ \hline
		recall   & 1,00   & 0.933  & 0.3   & 1.00   & 0.95   & 0.32  \\ \hline
		accuracy & 0.749  & 0.787  & 0.845 & 0.745  & 0.794  & 0.88  \\ \hline
	\end{tabular}
\end{table}

Построенные модели с помощью логистической регрессии и метода опорных векторов показывают очень хорошие результаты на кросс-валидации, как в метрике recall, так и в accuracy. Из рассмотренных алгоритмов для дальнейшего анализа будет использована логистическая регрессия, которая дает возможность быстро отобрать признаки по их значимости в определении правильно ответа: будет страна подвержена водным конфликтам или нет. Рассмотрим результаты логистической регрессии с l1 – регуляризацией, позволяющей отбрасывать малозначимые признаки. В табл. \ref{log} приведены веса признаков, выдаваемые логистической регрессией с l1-регуляризацией, обученной на 5 рандомных подвыборках стран, включающих в себя 80\% объектов. Также для этих алгоритмов подсчитан recall и accuracy.
\begin{table}[]
	\centering
	\caption{\, Лог. регрессия с l1 - регуляризацией}
	\label{log}
	\begin{tabular}{|l|l|l|l|}
		\hline
		& LR(1)  & LR(2)  & LR(3)  \\ \hline
		x1       & 0      & 0      & 0      \\ \hline
		x2       & -0.568 & -0.42  & -0.537 \\ \hline
		x3       & -0.018 & 0      & -0.014 \\ \hline
		x4       & 0.097  & 0      & 0.145  \\ \hline
		x5       & 0      & 0      & 0      \\ \hline
		x6       & 0.175  & 0.361  & 0.367  \\ \hline
		x7       & 1.142  & 1.121  & 1.311  \\ \hline
		x8       & 0      & 0      & 0      \\ \hline
		x9       & 0      & 0      & 0      \\ \hline
		x10      & 0      & 0.370  & 0.221  \\ \hline
		x11      & -0.068 & -0.122 & 0      \\ \hline
		x12      & 0      & 0      & 0      \\ \hline
		x13      & -0.713 & -0.715 & -0.309 \\ \hline
		x14      & 0      & 0      & 0      \\ \hline
		x15      & 0      & 0.168  & 0.106  \\ \hline
		recall   & 1.00   & 1.00   & 1.00   \\ \hline
		accuracy & 0.714  & 0.762  & 0.762  \\ \hline
	\end{tabular}
\end{table}
\subsection{Анализ полученных результатов и построение предсказаний}
Обучив классификатор на данных для 103 стран, можно поставить под сомнение важность некоторых причин возникновения конфликтов, предложенных в тех или иных работах. Это можно сделать из-за высоких значений метрик recall и accuracy, что дает возможность доверять результатам логистической регрессии. Во-первых, на подверженность межстрановым или межгрупповым водным конфликтам не влияет объем располагаемых водных ресурсов, которые получены за счет эндогенных осадков. Также на прямую не провоцируют водные конфликты большие объемы пресной воды, протекающие по границам нескольких стран, а также подземная вода, которая поступает из иностранных источников. Кроме того получилось, что несущественным признаком являются сильные межгодовые колебания в объеме располагаемых водных ресурсов, хотя для бассейна реки Нил эта причина отмечалась авторами в качестве одной из основных, ведущей к нестабильности в регионе. ВВП на душу населения на 2013 год, а также процент людей, имеющий доступ к воде, прошедшей первичную обработку, также для всей совокупности стран не являются существенными факторами. Неважным признаком оказалась и высокая плотность населения. Шоки, связанные со строительством новых дамб, а также введенный индекс кооперации, в нашей модели также не являются значимыми в определении того будет ли страна подвержена водным конфликтам или нет. 

В то же время отметим, что небольшая часть причин, из отмеченных при рассмотрении водных противоречий в отдельных регионах, объясняет большую часть возникающих водных противоречий. Самым важным признаком оказалась средняя доля пресной воды, потребленная сельскохозяйственным сектором от общего объема всей потребленной воды. Чем больше эта доля, тем вероятнее страна окажется подвержена водным конфликтам. Также оказался ожидаемым тот факт, что объем поверхностной воды, поступающий в некоторую страну из водных источников, которые протекают также по территории других стран, влияет на подверженность страны водным конфликтам. Однако зависимость получилась следующая: чем больше этот объем, тем менее вероятны конфликты между странами. Из этого факта, а также из малой значимости таких признаков, как <<коэффицент кооперации>>, <<объем воды, протекающей по приграничным рекам>>, можно заключить, что на подверженность водным конфликтам влияет сильнее не уровень кооперации между странами, расположенными в одном и том же бассейне, а физический объем воды, которым располагает население страны. Этот факт говорит в пользу высокой значимости именно физического объема пресной воды в поддержании стабильности в регионе. Кроме того важной причиной оказалась коррупция: чем меньше страна подвержена коррупции, тем ниже вероятность развития водных конфликтов в ней. 

Для построения модели, предсказывающей, какие страны в ближайшем будущем могут оказаться подверженными водным конфликтам, может быть использовано свойство компактности. Оно заключается в том, что значения признаков для стран одного класса похожи. Одним из алгоритмов, реализующих эту идею, явлется KDTree, который считает <<расстояние>> между значениями признаков и выдает страны с наиболее похожими данными по признакам. В силу свойства компактности есть основания предполагать, что, если некоторая страна будет замечена в числе первых 3 ближайших соседей для нескольких стран, уже подверженных на текущий момент водным конфликтам, то велика вероятность того, что эта страна окажется подверженной водным конфликтам в ближайшем будущем. Отметим, что близость признаков между странами была измерена по признакам, имеющим наибольшую значимость в определении подверженности регионов водным конфликтам на основе результатов, представленных в табл. \ref{log}: x2, x6, x7, x10, x13.

Поcтроенная предсказательная модель на основе данных о 103 странах показала, что 6 новых стран в ближайшем будущем могут оказаться подверженными водным конфликтам. Так, например, Эритрея встречается в числе 3 ближайших соседей для 2 стран, подверженных на текущий момент противоречиям относительно воды. Большой объем потребляемой пресной воды сельским хозяйством (82\% в среднем за 1990 – 2015 года)  в купе с низким объемом располагаемых водных ресурсов на душу населения и катастрофической коррупцией могут в скором времени привести к тому, что Эритрея окажется подверженной череде водных конфликтов. Несмотря на то, что Замбия располагает большими запасами воды на душу населения (более 8500 $m^3$ на душу населения), страна является одной из беднейших в мире, что несомненно повлияло на <<решение>> модели отнести Замбию в число стран с высокой вероятностью развития водных конфликтов. Кроме того высокую вероятность возникновения водных противоречий в Замбии определяет высокий спрос на пресную воду, предъявляемый сельским хозяйством, а также высокая коррупция. Властям Гаити также следует задуматься о изменении структуры спроса: на сельскохозяйственный сектор в стране приходится примерно 90\% всего потребления. Кроме того властям Гаити для предотвращения конфликтов необходимо разобраться с коррупцией, эффект от которой еще более сильный в условиях очень низких доходов населения. С теми же проблемами, что и в Гаити сталкиваются также власти Мавритании. И, наконец, несмотря на то, что на сельскохозяйственный сектор в Нигерии в среднем приходится небольшая доля затрачиваемых водных ресурсов (53,6\%), высокая коррупция, а также высокие темпы прироста ВВП и низкий объем располагаемых водных ресурсов на душу населения также поднимают проблему развития водных противоречий в стране в ближейшем будущем.
\section{Выводы}
Увеличение численности населения и изменение структуры потребления в развивающихся стран несомненно усилило конкуренцию за водные ресурсы в отдельных регионах мира. Ужесточившаяся конкуренция вылилась в последнее время в увеличившееся число водных конфликтов, зарегестрированных в различных регионах мира. В сложившейся ситуации необходим детальный анализ причин возникновения конфликтов, как для сведения на нет уже существующих в некоторых странах противоречий относительно водных ресурсов, так и для предупреждения развития водных конфликтов в странах, до этого не подверженных им.


Основной целью исследования было выявить наиболее значимые общие для всех стран факторы, определяющие подверженность водным конфликтам, а также определить, какие страны могут в ближайшем будущем пострадать от пагубных последствий противоречий вокруг водных ресурсов. Для достижения данной цели, работа была разделена на три этапа:
\begin{enumerate}
	\item Собрано и рассмотрено большое число причин, предлагаемых различными исследователями в качестве основных, определивших или наоборот предотвративших развитие конфликтов в отдельных странах или регионах в течение 1990 – 2016 годов.
	\item Собраны данные, количественно описывающие для различных стран найденные причины.
	\item С помощью машинного обучения выяснено, какие причины являются наиболее важными в определении подверженности стран водным конфликтам, а также получено, какие новые регионы вскоре скорее всего будут охвачены проблемой водных конфликтов.
\end{enumerate}
Приведем основные выводы, полученные в данной работе:
\begin{enumerate}
	\item Наиболее значимыми факторами в определении подверженности всех стран водным конфликтам являются в порядке убывания значимости: доля воды, потребляемой сельскохозяйственным сектором от общего объема потребления пресной воды в экономике; фактический запас водных ресурсов на душу населения; уровень коррупции.
	\item Намного менее значимыми причинами водных конфликтов, которые в некоторых работах все же считаются определяющими в развитии противоречий вокруг водных ресурсов, оказались: уровень кооперации, загрязненность вод, плотность населения и сильные межгодовые колебания уровня воды в основных реках страны.
	\item Факторы, обозначенные в пункте 1., а также среднегодовой темп прироста ВВП страны с 1990 – 2013 год и зависимость региона от водных ресурсов, которые берут свой исток в других странах, позволяют в среднем выявить на основе имеющихся данных почти все случаи потенциальных водных конфликтов, а также примерно в 75\% случаях не ошибиться, указывая на возможное развитие водных конфликтов в той или иной стране.
	\item В ближайшем будущем следующие страны вероятнее всего окажутся подверженными водным конфликтам: Эритрея, Замбия, Гаити, Нигерия, Мавритания.
\end{enumerate}

Анализ основных причин развития водных конфликтов в странах показывает, что ключевыми мерами, направленными на предотвращение водных конфликтов в странах должны стать: система тарифов и квот, снижающая спрос на пресную воду со стороны аграрного сектора; уменьшение уровня коррупции хотя бы связанной с водными ресурсами; снижение физического дефицита водных ресурсов за счет стимулирования инноваций, связанных с производством вторичной воды.
\newpage
\nocite{*}
\printbibliography[title={Литература и электронные источники}, heading=bibintoc]
\end{document}